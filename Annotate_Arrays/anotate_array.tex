% Created 2019-02-13 mié 15:21
% Intended LaTeX compiler: pdflatex
\documentclass[11pt]{article}
\usepackage[utf8]{inputenc}
\usepackage[T1]{fontenc}
\usepackage{graphicx}
\usepackage{grffile}
\usepackage{longtable}
\usepackage{wrapfig}
\usepackage{rotating}
\usepackage[normalem]{ulem}
\usepackage{amsmath}
\usepackage{textcomp}
\usepackage{amssymb}
\usepackage{capt-of}
\usepackage{hyperref}
\author{Lucas}
\date{\today}
\title{}
\hypersetup{
 pdfauthor={Lucas},
 pdftitle={},
 pdfkeywords={},
 pdfsubject={},
 pdfcreator={Emacs 25.3.2 (Org mode 9.2)},
 pdflang={English}}
\begin{document}

\tableofcontents

This is a script for annotating an array given it's probe-gene table and a gff.

\section{Code}
\label{sec:org22ee9ee}
\subsection{Create rosetta}
\label{sec:org7068ac6}
Rosetta is a dictionary with information of gene names and annotation.
It is created merging info from:
\begin{itemize}
\item The gff (from plasmoDB)
\item A file containing gene aliases (from PlasmoDB): \href{file:///home/lucas/ISGlobal/Gen\_Referencies/Gene\_references\_rosetta.txt}{aliases\textsubscript{file}}.
\item A file containing "names" of the genes: \href{file:///home/lucas/ISGlobal/Gen\_Referencies/gene\_names.txt}{gene\textsubscript{names}\textsubscript{file}}.
\end{itemize}

\begin{verbatim}
#!/usr/bin/env python

rosetta = {}
with open("/home/lucas/ISGlobal/Gen_Referencies/Gene_references_rosetta.txt", "r+") as file1:
    for line in file1:
        rosetta[str(line.split("\t")[0].strip())] = {
            "old_refs": line.split("\t")[1:]}
    for key, value in rosetta.items():
        value["old_refs"][-1] = value["old_refs"][-1].strip()

with open("/home/lucas/ISGlobal/Gen_Referencies/PlasmoDB-41_Pfalciparum3D7.gff", "r+") as file2:
    for line in file2:
        if line.startswith("#"):
            pass
        elif line.split()[2] == "gene":
            line_split = line.strip().split("\t")[8].split(";")
            if line_split[0].replace("ID=", "") in rosetta.keys():
                rosetta[line_split[0].replace("ID=", "")]["annot"] = line_split[1].replace(
                    "description=", "")
        else:
            pass

with open("/home/lucas/ISGlobal/Gen_Referencies/gene_names.txt", "r+") as file3:
    header = True
    for line in file3:
        if header:
            pass
            header = False
        else:
            if line.strip().split()[0] in rosetta.keys():
                if line.strip().split("\t")[4] == "N/A":
                    rosetta[line.strip().split("\t")[0]]["name"] = "NA"
                else:
                    rosetta[line.strip().split("\t")[0]
                            ]["name"] = line.strip().split("\t")[4]


\end{verbatim}

\subsection{Load array to gene mapping and status}
\label{sec:orgbf8bfab}
Load the description of the array:
\begin{itemize}
\item Probenames
\item Target gene for each probe
\item Whether it should be kept (we remove probes that map to multiple genes).
\item We add annotation for the new probes and GDV1
\end{itemize}

\begin{verbatim}
import collections as col
import re

array_dict = col.defaultdict(dict)
with open("/media/lucas/Disc4T/Projects/Microarrays_R_analysis/array_decription.csv") as infile:
    for line in infile:
        line_list = line.strip().split()
        probe = line_list[1]
        gene = line_list[3]
        status = line_list[4]

        array_dict[probe] = {"gene":gene, "status":status}

for k,v in array_dict.items():
    if k.startswith("PF3D7"):
        v["gene"] = re.sub(r'_n\d.*', "", k)

for k,v in array_dict.items():
    if k.startswith("gdv1"):
        v["gene"] = "PF3D7_0935400"
\end{verbatim}

\subsection{Load array info}
\label{sec:orga7c4747}
\begin{verbatim}
with open("/media/lucas/Disc4T/Projects/Oriol/Microarrays/Raw_Data/US10283823_258576310003_S01_GE2_1105_Oct12_1_1.txt", "r+") as infile:

    skip = 10
    i = 1

    for line in infile:
        if i > skip:

            probe = line.split()[6]
            gene = array_dict[probe]["gene"]
            status = array_dict[probe]["status"]

            try:
                name = rosetta[gene]["name"]
                anot = rosetta[gene]["annot"]
                print("\t".join([probe, gene, status, name, anot]))

            except:
                print("\t".join([probe, gene, status, gene, gene]))


        else:
            i += 1
\end{verbatim}
\end{document}
